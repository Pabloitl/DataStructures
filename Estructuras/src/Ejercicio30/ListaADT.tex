\documentclass{article}

\usepackage[margin = 1cm]{geometry}

\begin{document}
    \section*{Definición}

    \LARGE
    Matemáticamente, una lista es una secuencia de cero o más elementos
    de un determinado tipo.

    $$ (a_1, a_2, a_3, \dots, a_n) $$
    donde $ n \ge 0 $, si $ n = 0 $ la lista es vacía.

    Los elementos de la lista etienen la propiedad de que sus elementos están ordenados de forma lineal, según las posiciones que ocupan en la misma. Se dice que $ a_i $ precede a $ a_{i + 1} $ para $ i = 1, \dots, n - 1 $ y que $ a_i $ sucede a $ a_{i + 1} $ para $ i = 2, \dots, n $.

    Las siguiente operaciones se restringen con:
    $$ \forall L \in Lista, \forall x \in Lista, \forall p \in puntero $$

    \Large
    \section*{Constructores}

    \begin{center}
        \begin{tabular}{|l|l|}
            \hline
            listaVacia(L) & Inicializa la lista L como lista vacía \\
            \hline
        \end{tabular}
    \end{center}

    \section*{Getters}

    \begin{center}
        \begin{tabular}{|l|l|}
            \hline
            getAnterior(p) & Devuelva la posición/dirección del nodo anterior a p \\
            \hline
            getPrimero() & Devuelve la posición/dirección del primer nodo de la lista L \\
            \hline
        \end{tabular}
    \end{center}

    \section*{Operaciones}

    \begin{center}
        \begin{tabular}{|l|l|}
            \hline
            esVacia() & Determina si la lista L está vacía \\
            \hline
            insertar(x, p) & Inserta en la lista L un nodo con el campo dato x, \\&delante del nodo de dirección p \\
            \hline
            inserPrimero(x) & Inserta un nodo con el dato x como primer nodo de la lista L \\
            \hline
            localizar(x) & Devuelve la posición/dirección donde está el campo de información x \\
            \hline
            suprimir(x) & Elimina de la lista el nodo que contiene el dato x \\
            \hline
            anula() & Vacia la lista L \\
            \hline
        \end{tabular}
    \end{center}

\end{document}
